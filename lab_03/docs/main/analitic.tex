\chapter{Аналитическая часть}

В данном разделе рассмотрены алгоритмы нахождения заданного значения в массиве\cite{search_algs}.

\section{Описание алгоритмов}

Данные алгоритмы используются для поиска заданного значения в массиве.

\subsection{Алгоритм линейного поиска}

При линейном поиске перебираются все элементы массива. Это означает, что если искомое значение находится в начале, оно будет найдено быстрее, чем если бы оно располагалось в конце \cite{lin_search_alg}.


\subsection{Алгоритм нахождения с помощью бинарного поиска}

При использовании алгоритма бинарного поиска перебор всех элементов не требуется. Для его работы множество должно быть отсортировано. Задаются левая и правая границы поиска, после чего выбирается центральный элемент в этом диапазоне и сравнивается с искомым значением. Если искомое значение меньше центрального элемента, правая граница смещается влево от него. Если больше — левая граница сдвигается вправо. Этот процесс продолжается до тех пор, пока искомое значение не совпадёт с центральным элементом или пока область поиска не сузится до нуля \cite{bin_search_alg}.

\clearpage

\paragraph*{ВЫВОД} ${}$ \\

В данном разделе рассмотрены алгоритмы нахождения заданного значения в массиве.
