\chapter{Технологическая часть}

В данном разделе будут приведены требования к программному обеспечению,
реализация алгоритмов и средства реализации.

\section{Требования к программному обеспечению}

Входные данные: Массив и искомое значение;
Выходные данные: Индекс и количество потребованных сравнений.

\section{Средства реализации}

Для реализации был выбран язык программирования $Rust$ \cite{rust}. Выбор обусловлен наличием библиотеки
$plotly$ \cite{plotly} для построения графиков.

\section{Реализация алгоритмов}

В листингах \ref{lst:linear_search} - \ref{lst:bin_search} представлены реализации алгоритмов.

\begin{center}
\captionsetup{justification=raggedright,singlelinecheck=off}
\begin{lstlisting}[label=lst:linear_search,caption=Алгоритм нахождения объектов линейным поиском]
pub fn linear_search<T: PartialEq>(arr: &[T], target: &T) -> (Option<usize>, usize) {
    let mut iterations = 0;

    for (index, item) in arr.iter().enumerate() {
        iterations += 1;
        if item == target {
            return (Some(index), iterations);
        }
    }

    (None, iterations)
}
\end{lstlisting}
\end{center}

\begin{center}
\captionsetup{justification=raggedright,singlelinecheck=off}
\begin{lstlisting}[label=lst:bin_search,caption=Алгоритм нахождения объектов бинарным поиском]
pub fn binary_search<T: PartialOrd>(arr: &[T], target: &T) -> (Option<usize>, usize) {
    let mut low = 0;
    let mut high = arr.len() as isize - 1;
    let mut iterations = 0;

    while low <= high {
        iterations += 1;
        let mid = ((low + high) / 2) as usize;

        if &arr[mid] == target {
            return (Some(mid), iterations);
        } else if &arr[mid] < target {
            low = mid as isize + 1;
        } else {
            high = mid as isize - 1;
        }
    }

    (None, iterations)
}
\end{lstlisting}
\end{center}

\paragraph*{ВЫВОД} ${}$ \\

В данном разделе были реализованы алгоритмы поиска заданного значения в массиве линейным и двоичным поисками, рассмотрены средства реализации, 
предусмотрены требования к программному обеспечению.

\clearpage