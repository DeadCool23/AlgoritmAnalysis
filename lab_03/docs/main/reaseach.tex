\chapter{Исследовательская часть}

\section{Технические характеристики}
Характеристики используемого оборудования:
\begin{itemize}
    \item Операционная система --- Windows 11 Home \cite{windows}
    \item Память --- 16 Гб.
    \item Процессор --- 12th Gen Intel(R) Core(TM) i7-12700H @  2.30 ГГц \cite{intel}
\end{itemize}

\section{Оценка алгоритмов}
В данном разделе оценку трудоемкости алгоритма будем дана в терминах числа сравнений, которые понадобились
для нахождения ответа. Размер массива равен $1084$.

\medskip

Для алгоритма линейного поиска количество возможных исходов соответствует длине массива: $n$ исходов, если элемент присутствует в массиве, и ещё один исход, если элемента нет. В худшем случае потребуется выполнить $n$ сравнений, что происходит, если элемент отсутствует в массиве или находится в его конце. Линейный поиск может работать быстрее на отсортированном массиве, если искомый элемент находится на индексе, меньшем $\log_2(n)$. На рисунке \ref{fig:plt_1} представлена гистограмма, показывающая работу алгоритма линейного поиска.

\begin{figure}[h]
    \centering
    \includegraphics[width=0.8\textwidth]{images/plots/lin.eps}
    \caption{Гистограмма алгоритма линейного поиска}
    \label{fig:plt_1}
\end{figure}

\clearpage

Для алгоритма с двоичным поиском наибольшее количество сравнений не превышает $log_2(n)$ в худшем случае. На рисунке \ref{fig:plt_2} и \ref{fig:plt_3} представлены гистограммы алгоритма двоичного поиска. На рисунке \ref{fig:plt_3} изображена гистограмма алгоритма с двоичным поиском, где количество сравнений отсортировано по возрастанию.

\begin{figure}[h]
    \centering
    \includegraphics[width=0.8\textwidth]{images/plots/bin.eps}
    \caption{Гистограмма алгоритма с бинарным поиском}
    \label{fig:plt_2}
\end{figure}

\begin{figure}[h]
    \centering
    \includegraphics[width=0.8\textwidth]{images/plots/sort_bin.eps}
    \caption{Гистограмма алгоритма с бинарным поиском, отсортированная по количеству сравнений}
    \label{fig:plt_3}
\end{figure}

\clearpage

\section{Вывод}

При полном переборе количество сравнений для поиска значения в массиве увеличивается по мере роста индекса искомого элемента. В отличие от этого, в бинарном поиске число сравнений практически не зависит от положения элемента в начале или конце массива. Однако элемент будет найден быстрее, если он находится ближе к середине или в её четвертях. Линейный поиск может быть быстрее бинарного на отсортированном массиве, если искомый элемент расположен на индексе, меньшем $\log_2(n)$, где $n$ — размер массива. Например, при размере массива $1084$, линейный поиск будет эффективнее двоичного для индексов от $1$ до $10$.

Тем не менее, наиболее эффективным считается бинарный поиск, так как в худшем случае количество сравнений не превышает $log_2(n)$, тогда как при полном переборе оно может достигать $n$.