\begin{center}
    \textbf{ВВЕДЕНИЕ}
\end{center}
\addcontentsline{toc}{chapter}{ВВЕДЕНИЕ}

В современном программировании эффективная работа с данными является важным фактором для повышения производительности приложений. Одним из методов оптимизации выполнения задач выступает конвейерная обработка~\cite{pipline_proc}, при которой процесс обработки данных разбивается на последовательные этапы. Каждый этап выполняется автономно, что позволяет обрабатывать данные параллельно и рациональнее использовать системные ресурсы.

\textbf{Цель лабораторной работы}: разработка ПО, выполняющего скачивание страниц и парсинг рецептов с сайта \url{gastronom.ru} с помощью параллельных вычислений по конвейерному принципу.

Для достижения поставленной цели необходимо выполнить следующие задачи:
\begin{itemize}
    \item[---] описать входные и выходные данные;
    \item[---] разработать ПО, выполняющего парсинг рецептов в конвейере;
	\item[---] выполнить анализ характеристик разработанного ПО на данных о интервалах времени стадий обработки рецепта.
\end{itemize}