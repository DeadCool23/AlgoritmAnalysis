\chapter{Входные и выходные данные}

Входные данные: сигнал о завершении конвейерной обработки

Выходные данные: файл $recipes.db$ с базой данных $SQLite$ считанных рецептов.

\clearpage

\chapter{Преобразование данных}

В интерфейсе программы ожидается нажатие клавиши $Enter$. При запуске программа считывает $URL$ адреса рецептов с базовой страницы \url{gastronom.ru} и записывает их в очередь которая подается алгоритму на вход. Алгоритм организован в виде конвейера из трёх этапов, каждый из которых выполняется в отдельном потоке.

\begin{enumerate}
    \item Чтение $HTML$ контента с $URL$.
    \item Парсинг $HTML$ контента:
    \begin{itemize}
        \item[---] считывание данных о рецепте и запись их в структуру данных;
        \item[---] считывание новых $URL$ адресов рецептов;
        \item[---] проверка новых $URL$ адресов рецептов на повторение.
    \end{itemize}
    \item Запись рецепта в БД.
\end{enumerate}

\clearpage