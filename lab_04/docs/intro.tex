\begin{center}
    \textbf{ВВЕДЕНИЕ}
\end{center}
\addcontentsline{toc}{chapter}{ВВЕДЕНИЕ}

Многопоточность — это способность процессора или его отдельных ядер одновременно обрабатывать несколько задач или потоков, что позволяет более эффективно использовать вычислительные ресурсы. В отличие от процессов, разделяют общую память и ресурсы процесса, к которому принадлежат. Благодаря этому обмен данными между потоками осуществляется быстрее, что снижает накладные расходы и повышает производительность системы в многозадачных приложениях~\cite{threads}.

\textbf{Цель лабораторной работы}: Получить навык организации параллельных вычислений на основе нативных потоков и сравнить последовательные и параллельные вычисления с использованием нативных потоков.

Для достижения поставленной цели необходимо выполнить следующие задачи:
\begin{itemize}
    \item[---] описать входные и выходные данные;
    \item[---] реализовать последовательную и параллельную с использованием нативных потоков загрузку  $HTML$---страниц;
    \item[---] выполнить сравнительный анализ алгоритмов по времени выполнения в зависимости от количества загружаемых страниц.
\end{itemize}