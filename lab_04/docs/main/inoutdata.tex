\chapter{Входные и выходные данные}

Входные данные: 
\begin{itemize}
    \item[---] количество страниц для загрузки с веб-сайта \url{gastronom.ru};
    \item[---] режим работы --- последовательный или параллельный;
    \item[---] количество потоков в параллельном режиме.
\end{itemize}

Выходные данные: директория с файлами, которые содержат скачанные данные со страниц рецептов.

\clearpage

\chapter{Преобразование данных}

В интерфейсе программы вводится количество считываемых страниц, один из двух режимов: последовательный или параллельный, а также количество потоков, если выбран параллельный режим. Программа загружает $HTML$---контент страниц рецептов с веб-сайта \url{gastronom.ru}, а далее сохраняет загруженные страницы в формате $JSON$ в директорию $recipes\_data$. Также программа выводит в консоль сообщения о ходе выполнения и возможных ошибках.

\clearpage