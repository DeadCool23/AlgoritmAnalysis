\chapter{Тестирование}

В таблице~\ref{tbl:tests} представлены функциональные тесты для разработанного
программного обеспечения. Все тесты пройдены успешно.

\begin{table}[h!]
    \begin{center}
		\begin{threeparttable}
    \caption{Описание тестовых случаев}
    \captionsetup{justification=raggedright, singlelinecheck=false}
    \label{tbl:tests}
    \begin{tabular}{|c|p{5cm}|p{7cm}|c|}
        \hline
        \textbf{№} & \textbf{Входные данные} & \textbf{Ожидаемый результат} & \textbf{Результат теста} \\
        \hline
        1 & 5, S & Успешная загрузка 5 страниц, сохранение в $recipes$ & Пройден \\
        \hline
        2 & 5, P, 3 & Успешная загрузка 5 страниц 2 потоками, сохранение в $recipes$ & Пройден \\
        \hline
        3 & a, s & Неверный ввод числа рецептов & Пройден \\
        \hline
        4 & -1, s & Неверный ввод числа рецептов & Пройден \\
        \hline
        5 & 2, p, a & Неверный ввод числа потоков & Пройден \\
        \hline
        6 & 2, p, -1 & Неверный ввод числа потоков & Пройден \\
        \hline
        7 & Отключенное интернет-соединение & Вывод сообщений об ошибках при попытке загрузки страниц & Пройден \\
        \hline
    \end{tabular}
    \end{threeparttable}
    \end{center}
\end{table}

\clearpage