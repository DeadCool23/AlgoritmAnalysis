\chapter*{Введение}
\addcontentsline{toc}{chapter}{Введение}

Расстояние Левенштейна и Дамерау-Левентшейна --- это минимальное количество операций вставки одного символа, удаления одного символа, замены одного символа на другой и транспозиции (перестановки двух соседних символов) в случае расстояния Дамерау-Левенштейна, необходимых для превращения одной строки в другую [1]. Расстояние Левенштейна используется для:
\begin{itemize}
    \item исправления ошибок в словах;
    \item поиска дубликатов текстов;
    \item для сравнения генов, хромосом и белков в биоинформатике;
    \item сравнения текстовых файлос помощью утилиты diff.
\end{itemize}

\textbf{Цель лабораторной работы} --- сравнение алгоритмов расстояния Левенштейна и Дамерау-Левенштейна. Для достижения поставленной цели необходимо выполнить следующие задачи:

\begin{itemize}
	\item изучить расстояния Левенштейна и Дамерау-Левенштейна;
	\item реализовать указвнные алгоритмы поиска расстояний (два алгоритма в матричной версии и один из алгоритмов в рекурсивной версии)
	\item провести сравнительный анализ линейной и рекурсивной реализаций алгоритмов по затраченному процессорному времени и памяти на основе экспериментальных данных;
	\item описать и обосновать полученные результаты в отчете о выполненной лабораторной работе.
\end{itemize}    