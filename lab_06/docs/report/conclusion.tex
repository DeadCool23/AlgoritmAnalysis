\begin{center}
    \textbf{ЗАКЛЮЧЕНИЕ}
\end{center}
\addcontentsline{toc}{chapter}{ЗАКЛЮЧЕНИЕ}

В ходе работы были проанализированы временные и алгоритмические сложности муравьиного алгоритма и метода полного перебора. Также проведены замеры времени выполнения, выполнена параметризация муравьиного алгоритма, что позволило определить оптимальные параметры для набора данных, представленного в приложении А. Наилучшие параметры оказались следующими: $\alpha$ = 0.75, $\rho$=0.1 и количество дней — 200.
\vspace{5mm}

В ходе выполнения данной лабораторной работы были решены следующие задачи:

\begin{itemize}
	\item сформулирована задача коммивояжера;
    \item рассмотрены методы решения с использованием полного перебора и муравьиного алгоритма;
    \item реализованы указанные алгоритмы;
    \item проведен сравнительный анализ времени работы алгоритмов;
    \item выполнена параметризация для муравьиного алгоритма.
\end{itemize}