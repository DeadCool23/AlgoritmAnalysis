\chapter{Аналитическая часть}

В данном разделе будет сформулирована задача коммивояжера, а также будут рассмотрены 2 метода решения этой задачи: муравьиный алгоритм и алгоритм полного перебора.

\section{Формулировка задачи коммивояжера}

Пусть задан граф $G = (V, E)$, где $V$ — множество вершин ($|V| = n$), а $E$ — множество ребер ($|E| = m$). Каждое ребро $(i, j) \in E$ имеет длину $c_{ij}$, которая задается матрицей расстояний $C = \|c_{ij}\|$. Если между вершинами $i$ и $j$ нет ребра, соответствующий элемент матрицы считается равным бесконечности ($c_{ij} = \infty$)~\cite{com_info}.

Произвольное подмножество попарно несмежных ребер графа $G$ называется паросочетанием в $G$. Паросочетание $A \subset E(G)$ называется совершенным, если каждая вершина графа $G$ инцидентна единственному ребру из $А$. Произвольная совокупность простых попарно непересекающихся циклов в графе $G$, покрывающая все вершины графа $G$, называется 2-фактором в $G$. \textbf{Гамильтоновым циклом} в графе $G$ называется 2-фактор, состоящий из одного цикла~\cite{gamelton}.

Требуется найти гамильтонов цикл, то есть цикл, проходящий через каждую вершину графа ровно один раз и возвращающийся в начальную точку, минимальной длины.

\section{Алгоритм полного перебора}

Алгоритм полного перебора для решения задачи коммивояжера заключается в рассмотрении всех возможных маршрутов в графе с целью нахождения минимального. Суть этого метода состоит в последовательном переборе всех вариантов обхода городов с выбором оптимального маршрута. Однако число возможных маршрутов стремительно увеличивается с ростом количества городов $n$, так как сложность алгоритма составляет $n!$. Несмотря на то, что алгоритм полного перебора гарантирует точное решение задачи, его использование приводит к значительным временным затратам уже при сравнительно небольшом числе городов.

\section{Муравьиный алгоритм}

Муравьиный алгоритм --- это метод решения задачи коммивояжера, основанный на моделировании поведения муравьиной колонии~\cite{ants}.

Каждый муравей прокладывает маршрут, используя информацию о феромонах, оставленных другими муравьями на графе. В процессе движения муравей оставляет феромон на своем пути, чтобы другие могли ориентироваться на него. Постепенно феромоны на оптимальном маршруте накапливаются, так как он используется наиболее часто.

Характеристики муравья:
\begin{itemize}
    \item \textbf{зрение} — муравей способен оценивать длину ребер.
    \item \textbf{память} — запоминает посещенные вершины.
    \item \textbf{обоняние} — реагирует на феромоны, оставленные другими муравьями.
\end{itemize}

\subsubsection*{Целевая функция}

Для оценки привлекательности перехода используется функция видимости~\eqref{d_func}:

\begin{equation}
    \label{d_func}
    \eta_{ij} = \frac{1}{D_{ij}},
\end{equation}

где $D_{ij}$ — расстояние между вершинами $i$ и $j$.

\subsubsection*{Формула вероятности перехода}

Вероятность перехода муравья $k$ из текущей вершины $i$ в вершину $j$ рассчитывается по формуле~\eqref{posib}:

\begin{equation}
    \label{posib}
    P_{kij} = 
    \begin{cases}
        \frac{\tau_{ij}^a \eta_{ij}^b}{\sum_{q \in J_{ik}} \tau_{iq}^a \eta_{iq}^b}, & \text{если вершина $j$ еще не посещена муравьем $k$,} \\
        0, & \text{иначе,}
    \end{cases}
\end{equation}

где:
\begin{itemize}
    \item $a$ — параметр влияния феромона;
    \item $b$ — параметр влияния длины пути;
    \item $\tau_{ij}$ — количество феромонов на ребре $(i, j)$;
    \item $\eta_{ij}$ — видимость (обратная расстоянию).
\end{itemize}

\subsubsection*{Обновление феромонов}

После завершения движения всех муравьев уровень феромонов на ребрах обновляется по формуле~\eqref{update_phero_1}:

\begin{equation}
    \label{update_phero_1}
    \tau_{ij}(t+1) = (1-p)\tau_{ij}(t) + \Delta \tau_{ij},
\end{equation}

где $p$ — коэффициент испарения феромона, а $\Delta \tau_{ij}$ определяется как:

\begin{equation}
    \label{update_phero_2}
    \Delta \tau_{ij} = \sum_{k=1}^N \Delta \tau_{ij}^k,
\end{equation}

\begin{equation}
    \label{update_phero_3}
    \Delta \tau_{ij}^k = 
    \begin{cases}
        \frac{Q}{L_k}, & \text{если ребро $(i, j)$ посещено муравьем $k$,} \\
        0, & \text{иначе,}
    \end{cases}
\end{equation}

где $Q$ — параметр, связанный с длиной оптимального пути, а $L_k$ — длина маршрута муравья $k$.

\subsubsection*{Элитные муравьи}

Для улучшения временных характеристик муравьиного алгоритма вводят так называемых элитных муравьев. Элитный муравей усиливает ребра наилучшего маршрута, найденного с начала работы алгоритма. Количество феромона, откладываемого на ребрах наилучшего текущего маршрута $T^{+}$, принимается равным $Q / L^{+}$, где $L^{+}$ --- длина маршрута $T^{+}$. Этот феромон побуждает муравьев к исследованию решений, содержащих несколько ребер наилучшего на данный момент маршрута $T^{+}$. Если в муравейнике есть $e$ элитных муравьев, то ребра маршрута $T^{+}$ будут получать общее усиление:

\begin{equation}
    \label{elite_ant}
    \Delta \tau_{e} = e \cdot Q / L^{+}.
\end{equation}

\subsubsection*{Описание алгоритма}
\begin{enumerate}
    \item Муравей исключает из дальнейшего выбора вершины из список посещенных вершин, которые хранятся в памяти муравья (список запретов $J_{ik}$).
    \item Муравей оценивает привлекательность вершин на основе видимости, которая обратно пропорциональна расстоянию между вершинами.
    \item Муравей ощущает уровень феромонов на ребрах графа, который указывает на предпочтительность маршрута.
    \item После прохождения ребра $(i, j)$ муравей оставляет на нем феромон, причем его количество зависит от длины маршрута $L_k$, пройденного муравьем, и параметра $Q$.
\end{enumerate}

Таким образом, алгоритм постепенно находит оптимальный маршрут за счет коллективного взаимодействия муравьев и их способности к адаптации на основе накопленных феромонов.

\paragraph*{ВЫВОД} ${}$ \\

В результате аналитического раздела была представлена графовая формулировка задача коммивояжера, а также рассмотрены 2 метода ее решения: муравьиный алгоритм и алгоритм полного перебора.
