\begin{center}
    \textbf{ВВЕДЕНИЕ}
\end{center}
\addcontentsline{toc}{chapter}{ВВЕДЕНИЕ}

\textbf{Задача коммивояжера} — одна из наиболее известных и старейших задач комбинаторной оптимизации. В 1831 г. в Германии вышла книга под названием "Кто такой коммивояжер и что он должен делать для процветания своего предприятия".
Одна из рекомендаций этой книги гласила: "Важно посетить как можно больше мест возможного сбыта, не посещая ни одно из них дважды". Это была первая формулировка задачи коммивояжера~\cite{com_info}.

\textbf{Цель лабораторной работы} --- рассмотрение алгоритмов решения задачи коммивояжера.
Для достижения поставленной цели необходимо выполнить следующие задачи:

\begin{itemize}
	\item сформулировать задачу коммивояжера;
    \item рассмотреть методы решения с использованием полного перебора и муравьиного алгоритма;
    \item реализовать указанные алгоритмы;
    \item провести сравнительный анализ времени работы алгоритмов;
    \item выполнить параметризацию для муравьиного алгоритма.
\end{itemize}