\begin{center}
    \textbf{ЗАКЛЮЧЕНИЕ}
\end{center}
\addcontentsline{toc}{chapter}{ЗАКЛЮЧЕНИЕ}

Исследование показало, что классический алгоритм умножения матриц уступает по времени алгоритму Винограда примерно в 1.2 раза. Это связано с тем, что в алгоритме Винограда часть вычислений производится заранее, а количество сложных операций, таких как умножение, сокращается, что делает его более предпочтительным. Однако наилучшие результаты по скорости демонстрирует оптимизированный алгоритм Винограда, который на матрицах размером более 10 работает примерно в 1.3 раза быстрее, чем классический алгоритм. Это достигается за счет использования операций плюс-равно вместо равно и плюс, замены умножений сдвигами, а также предварительного вычисления некоторых элементов. Таким образом, для достижения максимальной производительности предпочтительнее использовать оптимизированный алгоритм Винограда.

\vspace{5mm}

В ходе выполнения данной лабораторной работы были решены следующие задачи:
В ходе выполнения данной лабораторной работы были решены следующие задачи:
\begin{itemize}
    \item[---] реализованны указанные алгоритмы;
    \item[---] выполна оценка трудоемкости разрабатываемых алгоритмов;
    \item[---] проведено сравнение требуемого времени выполнения алгоритмов;
    \item[---] описаны и обоснованы полученные результаты.
\end{itemize}